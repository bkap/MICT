\documentclass[12pt]{article}
\usepackage[utf8]{inputenc}
\usepackage{amsmath}
\title{Multi-user Infinite Canvas Thingy}
\begin{document}
 \maketitle
 \textbf{Client Specification}
 \begin{itemize}
 \item
    The root of the client program is the Client class. This is the application/applet itself. Other than initializing the other components and coordinating communication between them, no logic should be performed in this class.
 \item
     There is also a ClientState class. A single ClientState object will be initialized for each instance of the program. The ClientState object will be used to store the shared state. Any field that needs to be accessed and modifed by multiple classes should be placed in here. This includes but is not limited to the currently selected tool and color, the current location of the canvas, and the persistent connection to the server. The use of a singleton for shared state will reduce the coupling between all the components.
 \item
 The Client will have two components embedded within it: the Toolbox and the Canvas.
 \item
  The Toolbox will contain a series of Tools. Each Tool (which extends mict.tools.Tool) shall be defined in Jython so that we can serialize the class and transmit it from the server to the client. This way, every client will support every tool the server has and we don't have to worry about a client possessing a tool that the server doesn't know how to process.
  \begin{itemize}
  \item
  We will need to examine the various techniques for serializing classes to
  ensure that this is possible. Based on a preliminary analysis, this can be
  obtained with Java’s ObjectOutputStream. These tools
  will implement the functional requirements specified in section 3.2.2 of the
  Software requirements specification.

  \item
  Each tool will know how to serialize its commands so that we can send them to
  the server. They will also know their own name, associated icon, and tooltip.
  The tool needs to be entirely self-contained so the client doesn't need to
  know anything about how many tools there are, or what they do.   \item
  The interface specification for the Tool is attached.
  \end{itemize}
  \item
  For each Tool, we will make a ToolButton. ToolButton is a subclass of JButton. It has an ActionListener that sets the associated tool as the active tool within ClientState when the button is pressed.
  \item
  The Canvas shall be a blank JPanel.
  \begin{itemize}
      \item
      The Canvas shall have a single MouseListener. The MouseListener will wait for MousePressed, MouseDragged, and MouseReleased events and dispatch those events to the currently selected tool. Once the MouseReleasedEvent has been fired, the Canvas will request a serialized representation of the tool's action, and will dispatch it to the server.
  \end{itemize}
  \item
      There will be a single Java class and a single Python module (JythonBridge.java and javabridge.py) to coordinate communications between Jython and Java components. Any Java component that wants to access something from Jython should go through JythonBridge. Because of the way Jython works, Python components can access the Java libraries directly if they need to extend them (for instance, extending Tools) but if they wish to get an instance of a Java object, they should go through javabridge
  \item
     When the program is first launched, the Toolbox will be empty and the
     canvas will have 2 JTextFields, a JPasswordField, and a JButton. The first
     JTextfield will be used to enter the server to connect to. The second will
     be the username, and the JPasswordField will be used to enter the
     password. The JButton will tell the client to connect to the server
     specified with the given log-in credentials. If login is successful, the
     fields will be removed. This implements the Canvas.Connect requirement
     mentioned in section 3.1 of the SRS document.
\end{itemize}
\end{document}

