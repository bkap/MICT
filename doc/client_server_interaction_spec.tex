\documentclass[11pt,oneside,a4paper]{article}
\usepackage{textcomp}
\usepackage{setspace}
\usepackage{graphicx}
\title{Client Server Interactions - EECS 393}
\author{Don Huckle, Ben Kaplan, Mark Wyrzykowski, Rob Wiesler}
\begin{document}
\maketitle
\section{Client-Server Specification}
The client server interaction is based on a TCP connection between the server and the client.  
In the communication there are three major players.  
The Server's Waiter thread, The Client's client connection thread and the Client's GUI which will use event handlers such as mouse-clicked and mouse-dragged to handle relevant actions.  When the client is first run the GUI will be created.  


\subsection{Connecting to the Server}
The Client's GUI will have an event handler that creates the client connection that connects to the server when the login button is pushed. The connection to the server will be persistent;  Once established the waiter thread on the server's side and the Client Connection thread will communicate.  


\subsection{Client-Side Generated Communication}
When an event handler is called to do some operation it will often result in the event handler sending a message to the client connection thread. This thread will then pass the information on to the server. 
Operations that will cause communication to be generated by the client include: 
mouse clicked, mouse dragged, mouse released.  These events have event handlers associated with them.  The event handlers will send the positions of the mouse as well as the tool currently in use. Additionally data from client state may also be transmitted.  It is important to not that the tool object will be used to specify what information will be sent to the server.


\subsection{Server-Side Generated Communication}
When the Server has some action that it needs the client to take, such as correcting the client's canvas info so that it no longer includes an illegal draw operation(e.g. drawing on a protected part of the canvas); The server will initiate communication with the client.  When this happens the information is passed from the server to the client connection thread.  The client connection thread will send the information to the client class.  Because the Swing architecture is not thread safe the client class will have the GUI invoke the operation later by using the invokeLater(Runnable doRun) method.

\end{document}
