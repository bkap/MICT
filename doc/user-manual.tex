\documentclass[11pt,oneside,a4paper]{article}
\usepackage[utf8]{inputenc}
\usepackage{textcomp}
\usepackage{setspace}
\usepackage{graphicx}
\title{Multi-User Infinite Canvas Thing}
\author{Donald Huckle, Benjamin Kaplan, Mark Wyrzykowski, Rob Wiesler}
\begin{document}
\maketitle
\pagebreak
\tableofcontents
\pagebreak
\section{Welcome}
Congratulations on being able to figure out how to open this User Manual.
You're now well on your way to experiencing the revolutionary new image editor
known as the Multi-User Infinite Canvas Thingy, also called MICT. This program
creates a canvas that can be edited by multiple people at the same time
allowing for more productive days even when you are away from the office.

\section{System Requirements}
To use MICT to its full potential, you will need a client install and a server
install. These can be on the same machine, but that would be pointless.

\subsection{Client Requirements}
\begin{itemize}
\item A computer running Windows XP or newer, Mac OS X 10.5 or newer, or a
recent version of Linux (tested on Ubuntu 8.04 LTS and 10.04 LTS)
\item Java 1.6 or later. The program has been tested with the Oracle and Apple
JVMs. Other implementations of Java may not function properly
\item Apache Ant (tested with v. 1.8.1)
\item A broadband Internet connection
\item 15 MB of hard drive space
\item Some RAM, but so little that it's not worth discussing.
\item A graphics card. Any one manufactured in the last 15 years or so should
do

\end{itemize}
\subsection{Server System Requirements}
\begin{itemize}
\item A computer running a modern operating system. The server has only been
tested on Ubuntu 10.04 but should run on other versions of Linux, Mac OS X, and
Windows.
\item The PostgreSQL database (tested with 8.2)
\item Apache Ant (tested with v. 1.8.1) or GNU Make (Linux/Mac OS X only)
\item A broadband Internet connection
\item A processor
\item (optional, but preferred) a URL that it can be reached at
\end{itemize}

\section{Installing MICT}
First of all, please ensure that all prerequisites are installed on your
computer. Java and Apache Ant should be on your System path. If you're using
Linux or Mac OS X, It Just Works\texttrademark. If you are on Windows, you'll
have to work a little bit harder

TODO: Explain how to put stuff on path for Windows or find workaround

Simply extract the given compressed folder wherever you want the program- no
installation required.

\section{Launching MICT}
Double click on the included runclient script if you're running the client. If
you're on Mac OS X or Linux, click on the runclient.sh file. If you're on
Windows, use runclient.bat. For the server, follow the same instructions but
do runserver instead of run client.

\section{Connecting to a Server}
Upon launching the program, you'll be presented with a dialog asking you to
input a server name. Put the URL or IP address of the server you want to
connect to in the dialog box and press enter. The program will then load and
present you with the tools.

\section{Navigating the Program}
MICT's layout can be broken down into 3 parts. On the left, you have the
Toolbox. On the right, you have user controls. And in the middle, you have the
star of the show: the image itself.
\subsection{Toolbox}
The Toolbox has a very basic interface, similar to every other image editor out
there. It has a grid of buttons each with an icon representing what the tool
does. Hover over the button to see a more detailed description. Only one tool
can be selected at a time, and one tool is always selected. Beneath the tools
is a single button showing the currently selected color. All the tools that let
you select a color (which is most of them) will be drawn in that color. Click
on that button to change the active color.

\subsection{Canvas}
The canvas is where the magic actually happens. It shows a portion of the total
image, and updates any time someoone makes a change to your section of the
screen. By clicking and moving the mouse around the canvas, you can make your
own changes to the canvas.

Most of the tools are triggered by you pressing and dragging the mouse, then
releasing when you are done. There are only a couple tools that don't work this
way. First, you have the image tool. Click any location on the canvas and
you'll be pressented with a File dialog box. Find your image and press okay.
The image will be inserted into the canvas so that the top left corner of the
image is the spot where you clicked. The paste tool places the image in a
similar direction, although it will use the image taken with the copy tool
rather than prompting you for your own image. The text tool will prompt you for
a string to enter when you click.
\subsection{Admin Panel}
The Admin Panel has different actions available depending on your permissions.
You may have the ability to see all other users logged into the server, the
ability to kick other users off the server, and the ability to ban other users
permanently.


\section{Plug-ins}
MICT has a powerful plugin system that makes it easy to add new tools. Simply
place the plug-in script in the tools folder on the server and restart the
server. Upon connecting, every client will be given the new tool. To update a
tool, simply replace the old file with the new one. Once again, clients will be
updated automatically.

\subsection{Plugins for Developers}
Plugins are developed using the Python scripting language. They contain classes
which subclass mict.tools.Tool. See the developer documentation of that
interface for more information
\end{document}
